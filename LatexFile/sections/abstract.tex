\section*{Abstract}
Antimicrobial resistance (AMR), particularly multi-drug resistance (MDR), is a critical global health threat necessitating rapid and accurate diagnostic methods to supplement or replace time-consuming culture-based antimicrobial susceptibility testing (AST). Machine learning (ML) and deep learning (DL) models that predict AMR phenotypes from whole-genome sequencing (WGS) data offer a promising solution. This study evaluates and develops several advanced ML approaches to predict AMR across multiple bacterial species, focusing on improving performance, addressing data limitations, and enhancing model interpretability.
We applied a range of ML algorithms, including Random Forest (RF), Logistic Regression (LR), Support Vector Machine (SVM), Convolutional Neural Networks (CNN), and ensemble methods, to predict resistance phenotypes from genomic features such as single nucleotide polymorphisms (SNPs), k-mers, and gene content. A key challenge in AMR prediction is handling multi-drug resistance, where pathogens are resistant to multiple antibiotics simultaneously. Traditional models often predict resistance to single drugs in isolation. To address this, we developed a multi-label classification (MLC) approach using an Ensemble of Classifier Chains (ECC) model, which simultaneously predicts resistance to multiple drugs by accounting for correlations between them. When applied to 809 E. coli isolates, the ECC model significantly outperformed other MLC methods, demonstrating its potential for accurate MDR prediction.
Furthermore, ML models often struggle with novel antibiotics or imbalanced datasets where resistant samples are scarce. We demonstrate that deep transfer learning can overcome these limitations by transferring knowledge from a well-trained model (e.g., for an antibiotic with abundant data) to a new task with limited data. This approach significantly improved prediction performance for antibiotics with small, imbalanced datasets, highlighting its potential for rapid diagnostics of emerging resistances. To enhance predictive accuracy, we also developed a novel discriminative position-fused deep learning classifier that integrates an attention mechanism with positional features, enabling the model to focus on core SNPs critical for AMR. This attention-based model significantly outperformed traditional CNNs and other ML models, improving the average AUROC to 0.80 and the F1-score to 0.82.
Our findings show that ML models can achieve high predictive accuracy, with some models reaching 95–96% accuracy for predicting Minimum Inhibitory Concentrations (MICs) in nontyphoidal Salmonella. Feature analysis revealed that the models often identify known AMR genes, such as gyrA mutations for fluoroquinolone resistance and CMY-2 for β-lactam resistance, as primary drivers of resistance. The development of species-independent models, trained on data from multiple bacterial species, has also shown feasibility without compromising predictive power, paving the way for broader applications, including in metagenomics.
In conclusion, advanced ML techniques like multi-label classification, deep transfer learning, and attention-based neural networks offer robust and accurate solutions for predicting AMR from genomic data. These approaches can handle complex MDR patterns, overcome data scarcity for novel antibiotics, and provide insights into resistance mechanisms, thereby paving the way for faster clinical diagnostics and improved patient outcomes