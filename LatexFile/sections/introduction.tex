\section{Introduction}
Antimicrobial resistance (AMR) is a rapidly escalating global health crisis, representing one of the greatest threats to modern medicine. When disease-causing microorganisms like bacteria develop resistance to the drugs designed to treat them, the effectiveness of these treatments is nullified, leading to delayed patient recovery and increased mortality. According to World Health Organization (WHO) estimates, AMR could cause up to 10 million deaths annually by 2050, with an associated cost to healthcare systems of approximately 100 trillion. This threat is exacerbated by the overuse and misuse of antimicrobial drugs in clinical and agricultural settings, which fuels the selection and spread of resistant strains. A particularly serious concern is the rise of multi-drug resistance (MDR), where pathogens become resistant to multiple classes of antibiotics, leading to treatment failures and posing a significant challenge to public health. Escherichia coli, a major bacterial pathogen in clinical settings, is a prime example where the emergence of MDR has become a global health concern.
The gold standard for determining AMR in clinical practice is antimicrobial susceptibility testing (AST). However, traditional culture-based AST methods are time-consuming, often taking two days for common bacteria and several weeks for slow-growing organisms like Mycobacterium tuberculosis. This delay often forces clinicians to prescribe broad-spectrum empiric antibiotics, a practice that can contribute to the further spread of resistance. Consequently, there is an urgent need for rapid and accurate diagnostic methods to guide appropriate, targeted therapy.
Whole-genome sequencing (WGS) has emerged as a powerful alternative, offering a genetic-based approach to predict AMR phenotypes much faster than culture-based methods. Early genotypic prediction methods relied on rule-based approaches, which use prior knowledge to detect the presence or absence of known resistance genes or mutations. While effective for well-studied organisms and resistance mechanisms, these methods are difficult to scale, require constant manual curation, and struggle with complex or novel resistance patterns.
To overcome these limitations, machine learning (ML) and deep learning (DL) have been increasingly applied to predict AMR from WGS data. These data-driven approaches can learn the underlying patterns that connect genomic features—such as single nucleotide polymorphisms (SNPs), gene content, or k-mers (short DNA subsequences)—to resistance phenotypes without requiring a priori knowledge of the specific mechanisms involved. This allows ML models to not only predict resistance with high accuracy but also to potentially discover novel genetic determinants of AMR. Previous studies have demonstrated the high accuracy of ML models, such as Support Vector Machines (SVM), Logistic Regression (LR), Random Forest (RF), and Convolutional Neural Networks (CNNs), in predicting AMR for a variety of pathogens, including E. coli, Salmonella, and M. tuberculosis.
However, significant challenges remain in applying these models clinically. One major issue is the prediction of MDR, as most models focus on single-drug resistance and fail to account for the correlations and co-occurrence of resistance to multiple drugs. Another challenge is improving model performance for novel antibiotics or when faced with small, imbalanced datasets, which often hinder a model's accuracy and generalization. Furthermore, many complex DL models act as "black boxes," making it difficult to interpret the biological reasoning behind their predictions, a key requirement for clinical trust and application.
This paper addresses these gaps by exploring advanced ML methodologies for AMR prediction. We investigate the use of Multi-Label Classification (MLC) models, such as the Ensemble of Classifier Chains (ECC), to simultaneously predict resistance to multiple drugs in E. coli by modeling the dependencies between them. We also propose a discriminative position-fused deep learning classifier that incorporates an attention mechanism to enhance predictive performance by focusing on core SNPs critical to resistance. Finally, we demonstrate that deep transfer learning can significantly improve prediction accuracy for novel antibiotics or on small, imbalanced datasets by leveraging knowledge from well-trained models. By systematically evaluating these approaches, we aim to extend the available tools for AMR prediction, paving the way for improved diagnostics and more effective patient treatment in the fight against this global health threat.